\documentclass[ignorenonframetext,]{beamer}
\setbeamertemplate{caption}[numbered]
\setbeamertemplate{caption label separator}{: }
\setbeamercolor{caption name}{fg=normal text.fg}
\beamertemplatenavigationsymbolsempty
\usepackage{lmodern}
\usepackage{amssymb,amsmath}
\usepackage{ifxetex,ifluatex}
\usepackage{fixltx2e} % provides \textsubscript
\ifnum 0\ifxetex 1\fi\ifluatex 1\fi=0 % if pdftex
  \usepackage[T1]{fontenc}
  \usepackage[utf8]{inputenc}
\else % if luatex or xelatex
  \ifxetex
    \usepackage{mathspec}
  \else
    \usepackage{fontspec}
  \fi
  \defaultfontfeatures{Ligatures=TeX,Scale=MatchLowercase}
\fi
% use upquote if available, for straight quotes in verbatim environments
\IfFileExists{upquote.sty}{\usepackage{upquote}}{}
% use microtype if available
\IfFileExists{microtype.sty}{%
\usepackage{microtype}
\UseMicrotypeSet[protrusion]{basicmath} % disable protrusion for tt fonts
}{}
\newif\ifbibliography
\hypersetup{
            pdftitle={ggplot2をつかってみよう},
            pdfauthor={伊東宏樹},
            pdfborder={0 0 0},
            breaklinks=true}
\urlstyle{same}  % don't use monospace font for urls
\usepackage{color}
\usepackage{fancyvrb}
\newcommand{\VerbBar}{|}
\newcommand{\VERB}{\Verb[commandchars=\\\{\}]}
\DefineVerbatimEnvironment{Highlighting}{Verbatim}{commandchars=\\\{\}}
% Add ',fontsize=\small' for more characters per line
\usepackage{framed}
\definecolor{shadecolor}{RGB}{248,248,248}
\newenvironment{Shaded}{\begin{snugshade}}{\end{snugshade}}
\newcommand{\KeywordTok}[1]{\textcolor[rgb]{0.13,0.29,0.53}{\textbf{#1}}}
\newcommand{\DataTypeTok}[1]{\textcolor[rgb]{0.13,0.29,0.53}{#1}}
\newcommand{\DecValTok}[1]{\textcolor[rgb]{0.00,0.00,0.81}{#1}}
\newcommand{\BaseNTok}[1]{\textcolor[rgb]{0.00,0.00,0.81}{#1}}
\newcommand{\FloatTok}[1]{\textcolor[rgb]{0.00,0.00,0.81}{#1}}
\newcommand{\ConstantTok}[1]{\textcolor[rgb]{0.00,0.00,0.00}{#1}}
\newcommand{\CharTok}[1]{\textcolor[rgb]{0.31,0.60,0.02}{#1}}
\newcommand{\SpecialCharTok}[1]{\textcolor[rgb]{0.00,0.00,0.00}{#1}}
\newcommand{\StringTok}[1]{\textcolor[rgb]{0.31,0.60,0.02}{#1}}
\newcommand{\VerbatimStringTok}[1]{\textcolor[rgb]{0.31,0.60,0.02}{#1}}
\newcommand{\SpecialStringTok}[1]{\textcolor[rgb]{0.31,0.60,0.02}{#1}}
\newcommand{\ImportTok}[1]{#1}
\newcommand{\CommentTok}[1]{\textcolor[rgb]{0.56,0.35,0.01}{\textit{#1}}}
\newcommand{\DocumentationTok}[1]{\textcolor[rgb]{0.56,0.35,0.01}{\textbf{\textit{#1}}}}
\newcommand{\AnnotationTok}[1]{\textcolor[rgb]{0.56,0.35,0.01}{\textbf{\textit{#1}}}}
\newcommand{\CommentVarTok}[1]{\textcolor[rgb]{0.56,0.35,0.01}{\textbf{\textit{#1}}}}
\newcommand{\OtherTok}[1]{\textcolor[rgb]{0.56,0.35,0.01}{#1}}
\newcommand{\FunctionTok}[1]{\textcolor[rgb]{0.00,0.00,0.00}{#1}}
\newcommand{\VariableTok}[1]{\textcolor[rgb]{0.00,0.00,0.00}{#1}}
\newcommand{\ControlFlowTok}[1]{\textcolor[rgb]{0.13,0.29,0.53}{\textbf{#1}}}
\newcommand{\OperatorTok}[1]{\textcolor[rgb]{0.81,0.36,0.00}{\textbf{#1}}}
\newcommand{\BuiltInTok}[1]{#1}
\newcommand{\ExtensionTok}[1]{#1}
\newcommand{\PreprocessorTok}[1]{\textcolor[rgb]{0.56,0.35,0.01}{\textit{#1}}}
\newcommand{\AttributeTok}[1]{\textcolor[rgb]{0.77,0.63,0.00}{#1}}
\newcommand{\RegionMarkerTok}[1]{#1}
\newcommand{\InformationTok}[1]{\textcolor[rgb]{0.56,0.35,0.01}{\textbf{\textit{#1}}}}
\newcommand{\WarningTok}[1]{\textcolor[rgb]{0.56,0.35,0.01}{\textbf{\textit{#1}}}}
\newcommand{\AlertTok}[1]{\textcolor[rgb]{0.94,0.16,0.16}{#1}}
\newcommand{\ErrorTok}[1]{\textcolor[rgb]{0.64,0.00,0.00}{\textbf{#1}}}
\newcommand{\NormalTok}[1]{#1}

% Prevent slide breaks in the middle of a paragraph:
\widowpenalties 1 10000
\raggedbottom

\AtBeginPart{
  \let\insertpartnumber\relax
  \let\partname\relax
  \frame{\partpage}
}
\AtBeginSection{
  \ifbibliography
  \else
    \let\insertsectionnumber\relax
    \let\sectionname\relax
    \frame{\sectionpage}
  \fi
}
\AtBeginSubsection{
  \let\insertsubsectionnumber\relax
  \let\subsectionname\relax
  \frame{\subsectionpage}
}

\setlength{\parindent}{0pt}
\setlength{\parskip}{6pt plus 2pt minus 1pt}
\setlength{\emergencystretch}{3em}  % prevent overfull lines
\providecommand{\tightlist}{%
  \setlength{\itemsep}{0pt}\setlength{\parskip}{0pt}}
\setcounter{secnumdepth}{0}
%% PDFメタデータの文字化け防止
% https://blog.miz-ar.info/2015/09/latex-hyperref-tips/
% https://tex.stackexchange.com/questions/24445/hyperref-lualatex-and-unicode-bookmarks-issue-garbled-page-numbers-in-ar-for-l
\hypersetup{%
  pdfencoding=auto
}

%% Fonts
\usepackage[T1]{fontenc}
\usepackage{textcomp}
\usepackage[scale=1.0]{tgheros}
\usepackage[scaled]{beramono}

%% Japanese font
\usepackage{luatexja-otf}
\usepackage[match,deluxe,expert,noto-otf]{luatexja-preset}
\renewcommand{\kanjifamilydefault}{\gtdefault}

%% change fontsize of R code
% https://stackoverflow.com/questions/38323331/code-chunk-font-size-in-beamer-with-knitr-and-latex
\let\oldShaded\Shaded
\let\endoldShaded\endShaded
\renewenvironment{Shaded}{\footnotesize\oldShaded}{\endoldShaded}

%% change fontsize of output
\let\oldverbatim\verbatim
\let\endoldverbatim\endverbatim
\renewenvironment{verbatim}{\footnotesize\oldverbatim}{\endoldverbatim}

%% Title font size
\setbeamerfont{title}{size=\Huge, series=\bfseries}
\setbeamerfont{section title}{size=\LARGE, series=\bfseries}
\setbeamerfont{frametitle}{size=\Large, series=\bfseries}

\title{ggplot2をつかってみよう}
\author{伊東宏樹}
\date{2019/3/2}

\begin{document}
\frame{\titlepage}

\section{はじめに}

\begin{frame}{今回の発表に使用したコード・データ}

\url{https://github.com/ito4303/esj65} で公開

\end{frame}

\begin{frame}{このごろよくみるこのようなグラフ}

\includegraphics[height=7cm]{SappoRoR9_files/figure-beamer/introduction-1}

\end{frame}

\begin{frame}{ggplot2}

\LARGE

\begin{itemize}[<+->]
\tightlist
\item
  ggplot2パッケージ
\end{itemize}

\normalsize

\begin{itemize}[<+->]
\tightlist
\item
  gg: Grammer of Graphics
\end{itemize}

\large

\begin{itemize}[<+->]
\tightlist
\item
  統一された文法でさまざまな種類のグラフをえがける。
\item
  関連パッケージもどんどんできている。
\item
  GGally, ggmcmc, ggmap, ggthemes, egg\dots
\end{itemize}

\end{frame}

\begin{frame}{開発者など}

\begin{itemize}
\tightlist
\item
  開発者: Hadley Wickham \& Winston Chang
\item
  ウェブサイト: \url{http://ggplot2.tidyverse.org/}
\item
  開発履歴 \footnote{https://github.com/tidyverse/ggplot2/releases}

  \begin{itemize}
  \tightlist
  \item
    2015-01-10 ver. 1.0.0
  \item
    2015-12-19 ver. 2.0.0
  \item
    2016-03-02 ver. 2.1.0
  \item
    2016-11-15 ver. 2.2.0
  \item
    2016-12-31 ver. 2.2.1
  \end{itemize}
\end{itemize}

\end{frame}

\begin{frame}{ここでの目的}

\Large

\begin{itemize}
\item
  得られたデータを可視化する。
\item
  データにあった解析
\end{itemize}

\end{frame}

\begin{frame}{たとえば}

\includegraphics[width=10.5cm]{SappoRoR9_files/figure-beamer/r_eq_0-1}

どちらも\(r\)は0だが

\end{frame}

\begin{frame}{もうひとつの目的}

\begin{itemize}
\tightlist
\item
  グラフ作成に,手作業(コピー\&ペースト)をへらす。できれば
  なくす。\footnote{baseグラフィックでも可能ですが。}
\item
  R(と,RStudioなど)で自動化する。
\end{itemize}

\Large\bfseries

\begin{itemize}
\tightlist
\item
  手作業に由来するまちがいをなくす。
\item
  どういう処理をしたか,コードに保存する。
\item
  Reproducible Research(の一部)
\end{itemize}

\normalsize\mdseries

\begin{itemize}
\tightlist
\item
  この発表資料も,R Markdownで作成して,RStudio(とLaTeX)で一括処理。
\end{itemize}

\end{frame}

\section{ggplot2をつかってみる}\label{ggplot2}

\begin{frame}[fragile]{インストールとパッケージのロード}

\textbf{\texttt{ggplot2}}パッケージをインストールしておく。

\begin{Shaded}
\begin{Highlighting}[]
\KeywordTok{install.packages}\NormalTok{(}\StringTok{"ggplot2"}\NormalTok{)}
\end{Highlighting}
\end{Shaded}

パッケージをロードする。

\begin{Shaded}
\begin{Highlighting}[]
\KeywordTok{library}\NormalTok{(ggplot2)}
\end{Highlighting}
\end{Shaded}

あるいは,\textbf{\texttt{tidyverse}}パッケージをインストール,ロードしてもよい。

\vspace{12pt}

\textcolor{red}{\large\bfseries ggplot2はtidyデータと相性がいい。}

\end{frame}

\begin{frame}{使用するデータ}

京都市内の広葉樹二次林の毎木調査データ。\footnote{Itô (2017) Biodiversity Data Journal 5: e11732. \\
\hspace{14pt} https://doi.org/10.3897/BDJ.5.e11732 \\
\hspace{14pt} https://doi.org/10.5061/dryad.7f399 \\
\hspace{14pt} (CC0 https://creativecommons.org/publicdomain/zero/1.0/)}

50m \(\times\) 50mの範囲で,アラカシとクロバイの幹の座標(X,
Y)と胸高直径(DBH)を抽出する。

\end{frame}

\begin{frame}[fragile]{データの読み込みと整形}

1993年と2014年のデータと,2014年だけのデータを用意する。

\begin{Shaded}
\begin{Highlighting}[]
\KeywordTok{library}\NormalTok{(readr)}
\KeywordTok{library}\NormalTok{(dplyr)}

\NormalTok{github_dir <-}\StringTok{ }\KeywordTok{paste0}\NormalTok{(}\StringTok{"https://raw.githubusercontent.com/"}\NormalTok{,}
                     \StringTok{"ito4303/esj65/master/"}\NormalTok{)}
\NormalTok{s_data_file <-}\StringTok{ "Stem_data.csv"}
\NormalTok{m_data_file <-}\StringTok{ "Measurement_data.csv"}
\NormalTok{s_data <-}\StringTok{ }\KeywordTok{read_csv}\NormalTok{(}\KeywordTok{paste0}\NormalTok{(github_dir, s_data_file))}
\NormalTok{m_data <-}\StringTok{ }\KeywordTok{read_csv}\NormalTok{(}\KeywordTok{paste0}\NormalTok{(github_dir, m_data_file))}
\NormalTok{g_data <-}\StringTok{ }\KeywordTok{inner_join}\NormalTok{(m_data, s_data, }\DataTypeTok{by =} \StringTok{"Stem"}\NormalTok{) }\OperatorTok
\StringTok{  }\KeywordTok{filter}\NormalTok{(Year }\OperatorTok\StringTok{ }\KeywordTok{c}\NormalTok{(}\DecValTok{1993}\NormalTok{, }\DecValTok{2014}\NormalTok{) }\OperatorTok{&}\StringTok{ }\NormalTok{X1 }\OperatorTok{<}\StringTok{ }\DecValTok{50} \OperatorTok{&}
\StringTok{           }\NormalTok{Species }\OperatorTok\StringTok{ }\KeywordTok{c}\NormalTok{(}\StringTok{"Quercus glauca"}\NormalTok{,}
                          \StringTok{"Symplocos prunifolia"}\NormalTok{) }\OperatorTok{&}
\StringTok{           }\OperatorTok{!}\KeywordTok{is.na}\NormalTok{(DBH)) }\OperatorTok
\StringTok{  }\KeywordTok{transmute}\NormalTok{(}\DataTypeTok{X =}\NormalTok{ X1, }\DataTypeTok{Y =}\NormalTok{ Y1, Year, Species, DBH)}
\NormalTok{g_}\DecValTok{2014}\NormalTok{ <-}\StringTok{ }\KeywordTok{filter}\NormalTok{(g_data, Year }\OperatorTok{==}\StringTok{ }\DecValTok{2014}\NormalTok{)}
\end{Highlighting}
\end{Shaded}

\end{frame}

\begin{frame}[fragile]{データの確認}

\begin{Shaded}
\begin{Highlighting}[]
\KeywordTok{head}\NormalTok{(g_}\DecValTok{2014}\NormalTok{)}
\end{Highlighting}
\end{Shaded}

\begin{verbatim}
## # A tibble: 6 x 5
##       X     Y  Year Species                DBH
##   <dbl> <dbl> <dbl> <chr>                <dbl>
## 1   3.1   4    2014 Quercus glauca         3.8
## 2   2.8   1.4  2014 Symplocos prunifolia  19.2
## 3   0.1   7.3  2014 Symplocos prunifolia  17.9
## 4   2.8   6.7  2014 Symplocos prunifolia  15  
## 5   3.6   8.8  2014 Symplocos prunifolia  17.8
## 6   3.5   6.2  2014 Symplocos prunifolia  12.1
\end{verbatim}

このデータから,樹幹位置図を作成する。

\end{frame}

\begin{frame}[fragile]{ggplot関数}

\texttt{ggplot}オブジェクトを生成する。

\begin{Shaded}
\begin{Highlighting}[]
\NormalTok{p <-}\StringTok{ }\KeywordTok{ggplot}\NormalTok{(}\DataTypeTok{data =}\NormalTok{ g_}\DecValTok{2014}\NormalTok{)}
\KeywordTok{print}\NormalTok{(p)}
\end{Highlighting}
\end{Shaded}

\includegraphics[height=6cm]{SappoRoR9_files/figure-beamer/ggplot_func-1}

\end{frame}

\begin{frame}[fragile]{geom\_*関数}

\begin{itemize}
\tightlist
\item
  \textbf{\texttt{geom\_point}} 関数:
  散布図を描画(レイヤーに追加)する関数

  \begin{itemize}
  \tightlist
  \item
    \texttt{mapping} 引数: 変数のマッピングを渡す。
  \end{itemize}
\item
  \textbf{\texttt{aes}} 関数: 変数とグラフ要素とのaesthetic
  mappingを生成
\end{itemize}

\textcolor{red}{\bfseries \texttt{ggplot}オブジェクトに`+'演算子で,レイヤーを追加する。}

\begin{Shaded}
\begin{Highlighting}[]
\NormalTok{p }\OperatorTok{+}\StringTok{ }\KeywordTok{geom_point}\NormalTok{(}\DataTypeTok{mapping =} \KeywordTok{aes}\NormalTok{(}\DataTypeTok{x =}\NormalTok{ X, }\DataTypeTok{y =}\NormalTok{ Y))}
\end{Highlighting}
\end{Shaded}

\includegraphics[height=5cm]{SappoRoR9_files/figure-beamer/geom_func-1}

\end{frame}

\begin{frame}[fragile]{樹種ごとに色をかえる}

\textbf{\texttt{aes}}関数の\texttt{colour}(\texttt{color}でもよい)引数を指定する。

\begin{Shaded}
\begin{Highlighting}[]
\NormalTok{p }\OperatorTok{+}\StringTok{ }\KeywordTok{geom_point}\NormalTok{(}\DataTypeTok{mapping =} \KeywordTok{aes}\NormalTok{(}\DataTypeTok{x =}\NormalTok{ X, }\DataTypeTok{y =}\NormalTok{ Y, }\DataTypeTok{colour =}\NormalTok{ Species))}
\end{Highlighting}
\end{Shaded}

\includegraphics[height=5.5cm]{SappoRoR9_files/figure-beamer/colour-1}

\end{frame}

\begin{frame}[fragile]{こうしてもおなじ}

\begin{Shaded}
\begin{Highlighting}[]
\KeywordTok{ggplot}\NormalTok{(}\DataTypeTok{data =}\NormalTok{ g_}\DecValTok{2014}\NormalTok{,}
       \DataTypeTok{mapping =} \KeywordTok{aes}\NormalTok{(}\DataTypeTok{x =}\NormalTok{ X, }\DataTypeTok{y =}\NormalTok{ Y, }\DataTypeTok{colour =}\NormalTok{ Species)) }\OperatorTok{+}
\StringTok{  }\KeywordTok{geom_point}\NormalTok{()}
\end{Highlighting}
\end{Shaded}

\includegraphics[height=5.5cm]{SappoRoR9_files/figure-beamer/colour2-1}

\end{frame}

\begin{frame}[fragile]{すべての点の色をかえる}

\textbf{\texttt{aes}}関数の外で\texttt{colour}を指定する(\textbf{\texttt{geom\_point}}関数の\texttt{colour}引数に指定する)と,すべての点の色が指定した色になる。

\begin{Shaded}
\begin{Highlighting}[]
\NormalTok{p }\OperatorTok{+}\StringTok{ }\KeywordTok{geom_point}\NormalTok{(}\DataTypeTok{mapping =} \KeywordTok{aes}\NormalTok{(}\DataTypeTok{x =}\NormalTok{ X, }\DataTypeTok{y =}\NormalTok{ Y), }\DataTypeTok{colour =} \StringTok{"red"}\NormalTok{)}
\end{Highlighting}
\end{Shaded}

\includegraphics[height=5.5cm]{SappoRoR9_files/figure-beamer/colour3-1}

\end{frame}

\begin{frame}[fragile]{座標系}

Y軸は実は下むきなので,軸を反転させる(\textbf{\texttt{scale\_y\_reverse}}関数)。
また,X軸とY軸の長さをそろえる(\textbf{\texttt{coord\_fixed}}関数)。

\textcolor{red}{\bfseries 軸の指定なども`+'演算子で追加する。}

\begin{Shaded}
\begin{Highlighting}[]
\NormalTok{p2 <-}\StringTok{ }\KeywordTok{ggplot}\NormalTok{(g_}\DecValTok{2014}\NormalTok{) }\OperatorTok{+}\StringTok{ }\KeywordTok{xlim}\NormalTok{(}\DecValTok{0}\NormalTok{, }\DecValTok{50}\NormalTok{) }\OperatorTok{+}
\StringTok{  }\KeywordTok{scale_y_reverse}\NormalTok{(}\DataTypeTok{limits =} \KeywordTok{c}\NormalTok{(}\DecValTok{50}\NormalTok{, }\DecValTok{0}\NormalTok{)) }\OperatorTok{+}\StringTok{ }\KeywordTok{coord_fixed}\NormalTok{(}\DataTypeTok{ratio =} \DecValTok{1}\NormalTok{)}
\NormalTok{p2 }\OperatorTok{+}\StringTok{ }\KeywordTok{geom_point}\NormalTok{(}\DataTypeTok{mapping =} \KeywordTok{aes}\NormalTok{(}\DataTypeTok{x =}\NormalTok{ X, }\DataTypeTok{y =}\NormalTok{ Y, }\DataTypeTok{colour =}\NormalTok{ Species))}
\end{Highlighting}
\end{Shaded}

\includegraphics[height=5cm]{SappoRoR9_files/figure-beamer/coord-1}

\end{frame}

\begin{frame}[fragile]{樹種ごとに点の形をかえる}

\textbf{\texttt{aes}}関数の\texttt{shape}引数を指定する。

\begin{Shaded}
\begin{Highlighting}[]
\NormalTok{p2 }\OperatorTok{+}\StringTok{ }\KeywordTok{geom_point}\NormalTok{(}\KeywordTok{aes}\NormalTok{(}\DataTypeTok{x =}\NormalTok{ X, }\DataTypeTok{y =}\NormalTok{ Y,}
                    \DataTypeTok{colour =}\NormalTok{ Species, }\DataTypeTok{shape =}\NormalTok{ Species))}
\end{Highlighting}
\end{Shaded}

\includegraphics[height=5cm]{SappoRoR9_files/figure-beamer/point_shape-1}

\end{frame}

\begin{frame}[fragile]{点を大きくする}

\textbf{\texttt{geom\_point}}関数の\texttt{size}引数を指定する。

\begin{Shaded}
\begin{Highlighting}[]
\NormalTok{p2 }\OperatorTok{+}\StringTok{ }\KeywordTok{geom_point}\NormalTok{(}\KeywordTok{aes}\NormalTok{(}\DataTypeTok{x =}\NormalTok{ X, }\DataTypeTok{y =}\NormalTok{ Y,}
                    \DataTypeTok{colour =}\NormalTok{ Species, }\DataTypeTok{shape =}\NormalTok{ Species),}
                \DataTypeTok{size =} \DecValTok{4}\NormalTok{, }\DataTypeTok{alpha =} \FloatTok{0.7}\NormalTok{)}
\end{Highlighting}
\end{Shaded}

\includegraphics[height=5cm]{SappoRoR9_files/figure-beamer/point_size-1}

\end{frame}

\begin{frame}[fragile]{DBHに応じて点の大きさを変える}

\textbf{\texttt{aes}}関数の\texttt{size}引数を指定する。

\begin{Shaded}
\begin{Highlighting}[]
\NormalTok{p3 <-}\StringTok{ }\NormalTok{p2 }\OperatorTok{+}
\StringTok{  }\KeywordTok{geom_point}\NormalTok{(}\KeywordTok{aes}\NormalTok{(}\DataTypeTok{x =}\NormalTok{ X, }\DataTypeTok{y =}\NormalTok{ Y, }
                 \DataTypeTok{colour =}\NormalTok{ Species, }\DataTypeTok{shape =}\NormalTok{ Species, }\DataTypeTok{size =}\NormalTok{ DBH),}
             \DataTypeTok{alpha =} \FloatTok{0.7}\NormalTok{)}
\KeywordTok{print}\NormalTok{(p3)}
\end{Highlighting}
\end{Shaded}

\includegraphics[height=5cm]{SappoRoR9_files/figure-beamer/point_size2-1}

\end{frame}

\begin{frame}[fragile]{色を指定する}

\textbf{\texttt{scale\_colour\_manual}}関数で,任意の色を指定できる。

\begin{Shaded}
\begin{Highlighting}[]
\NormalTok{p3 }\OperatorTok{+}\StringTok{ }\KeywordTok{scale_colour_manual}\NormalTok{(}\DataTypeTok{values =} \KeywordTok{c}\NormalTok{(}\StringTok{"red"}\NormalTok{, }\StringTok{"#3355FF"}\NormalTok{))}
\end{Highlighting}
\end{Shaded}

\includegraphics[height=5cm]{SappoRoR9_files/figure-beamer/spec_colour-1}

\end{frame}

\begin{frame}[fragile]{漢字をつかう}

ここではIPAexゴシックフォントを使用した。

\begin{Shaded}
\begin{Highlighting}[]
\NormalTok{species <-}\StringTok{ }\KeywordTok{c}\NormalTok{(}\StringTok{"Quercus glauca"}\NormalTok{ =}\StringTok{ "アラカシ"}\NormalTok{,}
             \StringTok{"Symplocos prunifolia"}\NormalTok{ =}\StringTok{ "クロバイ"}\NormalTok{)}
\NormalTok{p4 <-}\StringTok{ }\NormalTok{p3 }\OperatorTok{+}\StringTok{ }\KeywordTok{labs}\NormalTok{(}\DataTypeTok{x =} \StringTok{"X軸 (m)"}\NormalTok{, }\DataTypeTok{y =} \StringTok{"Y軸 (m)"}\NormalTok{) }\OperatorTok{+}
\StringTok{  }\KeywordTok{scale_colour_discrete}\NormalTok{(}\DataTypeTok{name =} \StringTok{"種"}\NormalTok{, }\DataTypeTok{labels =}\NormalTok{ species) }\OperatorTok{+}
\StringTok{  }\KeywordTok{scale_shape_discrete}\NormalTok{(}\DataTypeTok{name =} \StringTok{"種"}\NormalTok{, }\DataTypeTok{labels =}\NormalTok{ species) }\OperatorTok{+}
\StringTok{  }\KeywordTok{scale_size_continuous}\NormalTok{(}\DataTypeTok{name =} \StringTok{"胸高直径 (cm)"}\NormalTok{) }\OperatorTok{+}
\StringTok{  }\KeywordTok{theme}\NormalTok{(}\DataTypeTok{text =} \KeywordTok{element_text}\NormalTok{(}\DataTypeTok{family =} \StringTok{"IPAexGothic"}\NormalTok{, }\DataTypeTok{size =} \DecValTok{10}\NormalTok{))}
\end{Highlighting}
\end{Shaded}

文字コードはUTF-8にしておくとよい。\footnote{日本語版WindowsのR.exeでは文字化けするかも。}

Windowsでは\textbf{\texttt{windowsFonts}}関数でフォントの設定をしておくか,\textbf{extrafont}パッケージをインストールして,必要な設定をしておく。

\end{frame}

\begin{frame}{漢字をつかった表示}

\includegraphics[height=7cm]{SappoRoR9_files/figure-beamer/Japanese_font2-1}

\end{frame}

\begin{frame}[fragile]{図をファイルに保存する}

\textbf{\texttt{ggsave}}関数などをつかう。
\footnote{R標準のグラフィックデバイス関数による出力もふつうにできる。}

\begin{Shaded}
\begin{Highlighting}[]
\KeywordTok{ggsave}\NormalTok{(}\StringTok{"p4.pdf"}\NormalTok{, }\DataTypeTok{device =}\NormalTok{ cairo_pdf,}
       \DataTypeTok{width =} \DecValTok{12}\NormalTok{, }\DataTypeTok{height =} \DecValTok{8}\NormalTok{, }\DataTypeTok{units =} \StringTok{"cm"}\NormalTok{)}
\end{Highlighting}
\end{Shaded}

日本語フォントを埋め込んだPDFを出力するには\\
\texttt{device\ =\ cairo\_pdf}\\
とする。\footnote{macOSではXQuartzのインストールが必要になるかもしれない。}
ただし,すべてのフォントでうまくいくとは限らない。

WindowsでPDFに日本語フォントを埋め込むためには,\textbf{extrafont}パッケージを利用する必要がある。
\footnote{ただし,OpenTypeフォントは利用不可らしい。}

\vspace{12pt}

macOSでは,\textbf{\texttt{ggsave}}関数のほか,\textbf{\texttt{quartz.save}}関数も利用可能。

\end{frame}

\begin{frame}[fragile]{themeをかえる(1)}

\begin{Shaded}
\begin{Highlighting}[]
\NormalTok{p4 }\OperatorTok{+}\StringTok{ }\KeywordTok{theme_bw}\NormalTok{(}\DataTypeTok{base_family =} \StringTok{"IPAexGothic"}\NormalTok{,}
              \DataTypeTok{base_size =} \DecValTok{12}\NormalTok{)}
\end{Highlighting}
\end{Shaded}

\includegraphics[height=6cm]{SappoRoR9_files/figure-beamer/theme_bw-1}

\end{frame}

\begin{frame}[fragile]{themeをかえる(2)}

\begin{Shaded}
\begin{Highlighting}[]
\NormalTok{p4 }\OperatorTok{+}\StringTok{ }\KeywordTok{theme_classic}\NormalTok{(}\DataTypeTok{base_family =} \StringTok{"NotoSansCJKjp-Regular"}\NormalTok{,}
                   \DataTypeTok{base_size =} \DecValTok{12}\NormalTok{)}
\end{Highlighting}
\end{Shaded}

\includegraphics[height=6cm]{SappoRoR9_files/figure-beamer/theme_classic-1}

\end{frame}

\begin{frame}[fragile]{facet: 分割して表示}

\textbf{\texttt{facet\_wrap}}関数を使用して,種ごとに分割して表示。

\begin{Shaded}
\begin{Highlighting}[]
\NormalTok{p4 }\OperatorTok{+}\StringTok{ }\KeywordTok{facet_wrap}\NormalTok{(}\OperatorTok{~}\StringTok{ }\NormalTok{Species) }\OperatorTok{+}\StringTok{ }\KeywordTok{theme}\NormalTok{(}\DataTypeTok{legend.position =} \StringTok{"none"}\NormalTok{)}
\end{Highlighting}
\end{Shaded}

\includegraphics[height=5.5cm]{SappoRoR9_files/figure-beamer/facet-1}

\end{frame}

\begin{frame}[fragile]{facet\_grid: 縦横に分割}

\begin{Shaded}
\begin{Highlighting}[]
\KeywordTok{ggplot}\NormalTok{(g_data, }\KeywordTok{aes}\NormalTok{(X, Y, }\DataTypeTok{colour =}\NormalTok{ Species, }\DataTypeTok{size =}\NormalTok{ DBH)) }\OperatorTok{+}
\StringTok{  }\KeywordTok{geom_point}\NormalTok{(}\DataTypeTok{alpha =} \FloatTok{0.7}\NormalTok{) }\OperatorTok{+}\StringTok{ }\KeywordTok{coord_fixed}\NormalTok{() }\OperatorTok{+}\StringTok{ }\KeywordTok{scale_y_reverse}\NormalTok{() }\OperatorTok{+}
\StringTok{  }\KeywordTok{scale_colour_discrete}\NormalTok{(}\DataTypeTok{guide =} \StringTok{"none"}\NormalTok{) }\OperatorTok{+}
\StringTok{  }\KeywordTok{facet_grid}\NormalTok{(Year }\OperatorTok{~}\StringTok{ }\NormalTok{Species)}
\end{Highlighting}
\end{Shaded}

\includegraphics[height=5.5cm]{SappoRoR9_files/figure-beamer/facet_grid-1}

\end{frame}

\begin{frame}[fragile]{ここまでのまとめ}

\Large

\begin{itemize}
\tightlist
\item
  \texttt{ggplot}関数でオブジェクト生成。
\item
  mapping引数に指定する\texttt{aes}関数で,データフレームのどのフィールドをどのグラフ要素に割り当てるかを指定する。
\item
  \texttt{geom\_*}関数で描画(レイヤー追加)。
\item
  テーマ,軸,Facetなどをそれぞれの関数で設定する。
\item
  パーツや設定を`+'演算子で追加していって,グラフを完成させる。
\end{itemize}

\end{frame}

\section{ほかの種類のグラフは}

\begin{frame}[fragile]{\textbf{\texttt{geom\_*}} 関数}

\begin{itemize}
\tightlist
\item
  \textbf{\texttt{geom\_bar}}
\item
  \textbf{\texttt{geom\_boxplot}}
\item
  \textbf{\texttt{geom\_contour}}
\item
  \textbf{\texttt{geom\_density}}
\item
  \textbf{\texttt{geom\_errorbar}}
\item
  \textbf{\texttt{geom\_histogram}}
\item
  \textbf{\texttt{geom\_line}}
\item
  \textbf{\texttt{geom\_polygon}}
\item
  \textbf{\texttt{geom\_smooth}}
\item
  \textbf{\texttt{geom\_ribbon}}
\item
  \textbf{\texttt{geom\_tile}}
\end{itemize}

などなど

\end{frame}

\begin{frame}[fragile]{次のデータ}

京都市内の広葉樹二次林で採取したアラカシの年輪データ。
\footnote{Itô and Sumida (2017) Ecol. Res. 32:105-105. \\
\hspace{14pt} https:doi.org/10.1007/s11284-016-1424-1 \\
\hspace{14pt} (CC-BY 4.0 https://creativecommons.org/licenses/by/4.0/)}

地際の各年の直径(mm)を使用。

\begin{Shaded}
\begin{Highlighting}[]
\NormalTok{q_data_file <-}\StringTok{ "Tree_ring_data.csv"}
\NormalTok{q_data <-}\StringTok{ }\KeywordTok{read_csv}\NormalTok{(}\KeywordTok{paste0}\NormalTok{(github_dir, q_data_file)) }\OperatorTok
\StringTok{  }\KeywordTok{filter}\NormalTok{(Height }\OperatorTok{==}\StringTok{ }\DecValTok{0}\NormalTok{) }\OperatorTok
\StringTok{  }\KeywordTok{mutate}\NormalTok{(}\DataTypeTok{No =} \KeywordTok{factor}\NormalTok{(No),}
         \DataTypeTok{Year =} \KeywordTok{as.integer}\NormalTok{(}\DecValTok{1994} \OperatorTok{-}\StringTok{ }\NormalTok{Year }\OperatorTok{+}\StringTok{ }\DecValTok{1}\NormalTok{)) }\OperatorTok
\StringTok{  }\KeywordTok{select}\NormalTok{(No, Year, Diameter)}
\end{Highlighting}
\end{Shaded}

\end{frame}

\begin{frame}[fragile]{データの確認}

\begin{Shaded}
\begin{Highlighting}[]
\KeywordTok{head}\NormalTok{(q_data)}
\end{Highlighting}
\end{Shaded}

\begin{verbatim}
## # A tibble: 6 x 3
##   No     Year Diameter
##   <fct> <int>    <dbl>
## 1 3      1994     29.2
## 2 3      1993     24.8
## 3 3      1992     21.8
## 4 3      1991     18.3
## 5 3      1990     16.2
## 6 3      1989     14.3
\end{verbatim}

\end{frame}

\begin{frame}[fragile]{折れ線グラフ}

\textbf{\texttt{geom\_line}}関数

No.3の幹のみを抽出して表示。

\begin{Shaded}
\begin{Highlighting}[]
\NormalTok{p <-}\StringTok{ }\KeywordTok{filter}\NormalTok{(q_data, No }\OperatorTok{==}\StringTok{ }\DecValTok{3}\NormalTok{) }\OperatorTok\StringTok{ }\KeywordTok{ggplot}\NormalTok{(}\KeywordTok{aes}\NormalTok{(Year, Diameter))}
\NormalTok{p }\OperatorTok{+}\StringTok{ }\KeywordTok{geom_line}\NormalTok{()}
\end{Highlighting}
\end{Shaded}

\includegraphics[height=5.5cm]{SappoRoR9_files/figure-beamer/geom_line1-1}

\end{frame}

\begin{frame}[fragile]{折れ線グラフ+点}

\textbf{\texttt{geom\_line}}と\textbf{\texttt{geom\_point}}の両方を使用。

\begin{Shaded}
\begin{Highlighting}[]
\NormalTok{p }\OperatorTok{+}\StringTok{ }\KeywordTok{geom_line}\NormalTok{(}\DataTypeTok{linetype =} \DecValTok{2}\NormalTok{) }\OperatorTok{+}\StringTok{ }\KeywordTok{geom_point}\NormalTok{(}\DataTypeTok{size =} \DecValTok{2}\NormalTok{)}
\end{Highlighting}
\end{Shaded}

\includegraphics[height=5.5cm]{SappoRoR9_files/figure-beamer/geom_line2-1}

\end{frame}

\begin{frame}[fragile]{データ全体をgeom\_lineで表示}

\begin{Shaded}
\begin{Highlighting}[]
\KeywordTok{ggplot}\NormalTok{(q_data) }\OperatorTok{+}\StringTok{ }\KeywordTok{geom_line}\NormalTok{(}\KeywordTok{aes}\NormalTok{(Year, Diameter))}
\end{Highlighting}
\end{Shaded}

\includegraphics[height=5.5cm]{SappoRoR9_files/figure-beamer/geom_line3-1}

うまくいかない。

\end{frame}

\begin{frame}[fragile]{groupを指定}

\begin{Shaded}
\begin{Highlighting}[]
\KeywordTok{ggplot}\NormalTok{(q_data) }\OperatorTok{+}\StringTok{ }\KeywordTok{geom_line}\NormalTok{(}\KeywordTok{aes}\NormalTok{(Year, Diameter, }\DataTypeTok{group =}\NormalTok{ No))}
\end{Highlighting}
\end{Shaded}

\includegraphics[height=5.5cm]{SappoRoR9_files/figure-beamer/geom_line5-1}

\end{frame}

\begin{frame}[fragile]{colourを指定}

\begin{Shaded}
\begin{Highlighting}[]
\KeywordTok{ggplot}\NormalTok{(q_data) }\OperatorTok{+}\StringTok{ }\KeywordTok{geom_line}\NormalTok{(}\KeywordTok{aes}\NormalTok{(Year, Diameter, }\DataTypeTok{colour =}\NormalTok{ No)) }\OperatorTok{+}
\StringTok{   }\KeywordTok{guides}\NormalTok{(}\DataTypeTok{colour =} \KeywordTok{guide_legend}\NormalTok{(}\DataTypeTok{ncol =} \DecValTok{2}\NormalTok{))}
\end{Highlighting}
\end{Shaded}

\includegraphics[height=5.5cm]{SappoRoR9_files/figure-beamer/geom_line4-1}

\end{frame}

\begin{frame}[fragile]{加工したプロットを重ねる}

アラカシNo.21の各年の肥大成長量を用意する。

\begin{Shaded}
\begin{Highlighting}[]
\NormalTok{q21 <-}\StringTok{ }\KeywordTok{filter}\NormalTok{(q_data, No }\OperatorTok{==}\StringTok{ }\DecValTok{21}\NormalTok{) }\OperatorTok
\StringTok{  }\KeywordTok{mutate}\NormalTok{(}\DataTypeTok{Growth =}\NormalTok{ Diameter }\OperatorTok{-}\StringTok{ }\KeywordTok{lead}\NormalTok{(Diameter)) }\OperatorTok
\StringTok{  }\KeywordTok{filter}\NormalTok{(}\OperatorTok{!}\KeywordTok{is.na}\NormalTok{(Growth)) }\OperatorTok\StringTok{ }\KeywordTok{arrange}\NormalTok{(Year)}
\NormalTok{p1 <-}\StringTok{ }\KeywordTok{ggplot}\NormalTok{(q21) }\OperatorTok{+}\StringTok{ }\KeywordTok{geom_line}\NormalTok{(}\KeywordTok{aes}\NormalTok{(Year, Growth))}
\KeywordTok{print}\NormalTok{(p1)}
\end{Highlighting}
\end{Shaded}

\includegraphics[height=4.5cm]{SappoRoR9_files/figure-beamer/q_data3-1}

\end{frame}

\begin{frame}[fragile]{平滑化した曲線を重ねる}

\textbf{\texttt{geom\_smooth}}関数で平滑化できる。\\
\textbf{\texttt{geom\_smooth}}\texttt{(method\ =\ "lm")}とすると直線回帰。

\begin{Shaded}
\begin{Highlighting}[]
\NormalTok{p1 }\OperatorTok{+}\StringTok{ }\KeywordTok{geom_smooth}\NormalTok{(}\KeywordTok{aes}\NormalTok{(Year, Growth), }\DataTypeTok{method =} \StringTok{"loess"}\NormalTok{)}
\end{Highlighting}
\end{Shaded}

\includegraphics[height=5.5cm]{SappoRoR9_files/figure-beamer/geom_smooth-1}

\end{frame}

\begin{frame}[fragile]{別のデータフレームから(1)}

dlmパッケージを使用して,カルマンフィルタでフィルタリングした値と95\%信頼区間をデータフレーム\texttt{kf}に格納する。

\begin{Shaded}
\begin{Highlighting}[]
\KeywordTok{library}\NormalTok{(dlm)}
\NormalTok{build <-}\StringTok{ }\ControlFlowTok{function}\NormalTok{(theta) \{}
  \KeywordTok{dlmModPoly}\NormalTok{(}\DataTypeTok{order =} \DecValTok{1}\NormalTok{, }\DataTypeTok{dV =}\NormalTok{ theta[}\DecValTok{1}\NormalTok{], }\DataTypeTok{dW =}\NormalTok{ theta[}\DecValTok{2}\NormalTok{])}
\NormalTok{\}}
\NormalTok{fit <-}\StringTok{ }\KeywordTok{dlmMLE}\NormalTok{(q21}\OperatorTok{$}\NormalTok{Growth, }\DataTypeTok{parm =} \KeywordTok{c}\NormalTok{(}\DecValTok{1}\NormalTok{, }\DecValTok{1}\NormalTok{), }
              \DataTypeTok{build =}\NormalTok{ build, }\DataTypeTok{lower =} \KeywordTok{c}\NormalTok{(}\DecValTok{0}\NormalTok{, }\DecValTok{0}\NormalTok{))}
\NormalTok{model <-}\StringTok{ }\KeywordTok{dlmModPoly}\NormalTok{(}\DataTypeTok{order =} \DecValTok{1}\NormalTok{,}
                    \DataTypeTok{dV =}\NormalTok{ fit}\OperatorTok{$}\NormalTok{par[}\DecValTok{1}\NormalTok{], }\DataTypeTok{dW =}\NormalTok{ fit}\OperatorTok{$}\NormalTok{par[}\DecValTok{2}\NormalTok{])}
\NormalTok{filter <-}\StringTok{ }\KeywordTok{dlmFilter}\NormalTok{(q21}\OperatorTok{$}\NormalTok{Growth, model)}
\NormalTok{kf <-}\StringTok{ }\KeywordTok{data.frame}\NormalTok{(}\DataTypeTok{Year =}\NormalTok{ q21}\OperatorTok{$}\NormalTok{Year,}
                 \DataTypeTok{Filter =} \KeywordTok{dropFirst}\NormalTok{(filter}\OperatorTok{$}\NormalTok{m)) }\OperatorTok
\StringTok{  }\KeywordTok{mutate}\NormalTok{(}\DataTypeTok{sd =} \KeywordTok{dlmSvd2var}\NormalTok{(filter}\OperatorTok{$}\NormalTok{U.C, filter}\OperatorTok{$}\NormalTok{D.C) }\OperatorTok
\StringTok{           }\KeywordTok{unlist}\NormalTok{() }\OperatorTok
\StringTok{           }\KeywordTok{dropFirst}\NormalTok{()) }\OperatorTok
\StringTok{  }\KeywordTok{mutate}\NormalTok{(}\DataTypeTok{Lower =} \KeywordTok{qnorm}\NormalTok{(}\FloatTok{0.025}\NormalTok{, Filter, sd),}
         \DataTypeTok{Upper =} \KeywordTok{qnorm}\NormalTok{(}\FloatTok{0.975}\NormalTok{, Filter, sd))}
\end{Highlighting}
\end{Shaded}

\end{frame}

\begin{frame}[fragile]{別のデータフレームから(2)}

カルマンフィルタでフィルタリングした値(赤線)と95\%信頼区間(うすい赤色の領域)を重ねて
表示する。

\begin{Shaded}
\begin{Highlighting}[]
\NormalTok{p1 }\OperatorTok{+}\StringTok{ }\KeywordTok{geom_line}\NormalTok{(}\DataTypeTok{data =}\NormalTok{ kf, }\KeywordTok{aes}\NormalTok{(Year, Filter), }\DataTypeTok{colour =} \StringTok{"red"}\NormalTok{) }\OperatorTok{+}
\StringTok{  }\KeywordTok{geom_ribbon}\NormalTok{(}\DataTypeTok{data =}\NormalTok{ kf, }\KeywordTok{aes}\NormalTok{(Year, }\DataTypeTok{ymin =}\NormalTok{ Lower, }\DataTypeTok{ymax =}\NormalTok{ Upper),}
              \DataTypeTok{fill =} \StringTok{"red"}\NormalTok{, }\DataTypeTok{alpha =} \FloatTok{0.3}\NormalTok{)}
\end{Highlighting}
\end{Shaded}

\includegraphics[height=4.5cm]{SappoRoR9_files/figure-beamer/geom_ribbon-1}

\end{frame}

\begin{frame}[fragile]{箱ひげ図}

アヤメのデータ \texttt{data(iris)} を使用。

\textbf{\texttt{geom\_boxplot}}関数

\begin{Shaded}
\begin{Highlighting}[]
\KeywordTok{ggplot}\NormalTok{(iris) }\OperatorTok{+}\StringTok{ }\KeywordTok{geom_boxplot}\NormalTok{(}\KeywordTok{aes}\NormalTok{(Species, Sepal.Width))}
\end{Highlighting}
\end{Shaded}

\includegraphics[height=5.5cm]{SappoRoR9_files/figure-beamer/boxplot-1}

\end{frame}

\begin{frame}[fragile]{X軸とY軸をいれかえる}

\textbf{\texttt{coor\_flip}}関数

\begin{Shaded}
\begin{Highlighting}[]
\KeywordTok{ggplot}\NormalTok{(iris) }\OperatorTok{+}\StringTok{ }\KeywordTok{geom_boxplot}\NormalTok{(}\KeywordTok{aes}\NormalTok{(Species, Sepal.Width)) }\OperatorTok{+}
\StringTok{  }\KeywordTok{coord_flip}\NormalTok{()}
\end{Highlighting}
\end{Shaded}

\includegraphics[height=5.5cm]{SappoRoR9_files/figure-beamer/boxplot2-1}

\end{frame}

\begin{frame}[fragile]{ヒストグラム(1)}

\textbf{\texttt{geom\_histogram}}関数

\texttt{fill}引数で塗りつぶしの指定。

\begin{Shaded}
\begin{Highlighting}[]
\KeywordTok{ggplot}\NormalTok{(iris) }\OperatorTok{+}
\StringTok{  }\KeywordTok{geom_histogram}\NormalTok{(}\KeywordTok{aes}\NormalTok{(Sepal.Length, }\DataTypeTok{fill =}\NormalTok{ Species),}
                 \DataTypeTok{binwidth =} \FloatTok{0.5}\NormalTok{)}
\end{Highlighting}
\end{Shaded}

\includegraphics[height=5cm]{SappoRoR9_files/figure-beamer/histogram-1}

\end{frame}

\begin{frame}[fragile]{ヒストグラム(2)}

\texttt{colour}と\texttt{fill}を別に指定。

\begin{Shaded}
\begin{Highlighting}[]
\KeywordTok{ggplot}\NormalTok{(iris) }\OperatorTok{+}
\StringTok{  }\KeywordTok{geom_histogram}\NormalTok{(}\KeywordTok{aes}\NormalTok{(Sepal.Length, }\DataTypeTok{fill =}\NormalTok{ Species),}
                 \DataTypeTok{colour =} \StringTok{"grey10"}\NormalTok{, }\DataTypeTok{binwidth =} \FloatTok{0.5}\NormalTok{)}
\end{Highlighting}
\end{Shaded}

\includegraphics[height=5cm]{SappoRoR9_files/figure-beamer/histogram2-1}

\end{frame}

\begin{frame}[fragile]{ヒストグラム(3)}

\texttt{position\ =\ "dodge"}: 積み重ねずに ずらす。

\begin{Shaded}
\begin{Highlighting}[]
\KeywordTok{ggplot}\NormalTok{(iris) }\OperatorTok{+}
\StringTok{  }\KeywordTok{geom_histogram}\NormalTok{(}\KeywordTok{aes}\NormalTok{(Sepal.Length, }\DataTypeTok{fill =}\NormalTok{ Species),}
                 \DataTypeTok{binwidth =} \FloatTok{0.5}\NormalTok{, }\DataTypeTok{position =} \StringTok{"dodge"}\NormalTok{)}
\end{Highlighting}
\end{Shaded}

\includegraphics[height=5.5cm]{SappoRoR9_files/figure-beamer/histogram3-1}

\end{frame}

\begin{frame}[fragile]{密度}

\textbf{\texttt{geom\_density}}関数

\begin{Shaded}
\begin{Highlighting}[]
\NormalTok{p <-}\StringTok{ }\KeywordTok{ggplot}\NormalTok{(iris, }\KeywordTok{aes}\NormalTok{(Sepal.Length, }\DataTypeTok{fill =}\NormalTok{ Species))}
\NormalTok{p }\OperatorTok{+}\StringTok{ }\KeywordTok{geom_density}\NormalTok{(}\DataTypeTok{alpha =} \FloatTok{0.5}\NormalTok{)}
\end{Highlighting}
\end{Shaded}

\includegraphics[height=5.5cm]{SappoRoR9_files/figure-beamer/geom_density-1}

\end{frame}

\begin{frame}[fragile]{注釈をつける}

\textbf{\texttt{annotate}}関数

\begin{Shaded}
\begin{Highlighting}[]
\NormalTok{x_pos <-}\StringTok{ }\KeywordTok{c}\NormalTok{(}\FloatTok{5.0}\NormalTok{, }\FloatTok{5.7}\NormalTok{, }\FloatTok{6.5}\NormalTok{); y_pos <-}\StringTok{ }\KeywordTok{c}\NormalTok{(}\FloatTok{0.6}\NormalTok{, }\FloatTok{0.8}\NormalTok{, }\FloatTok{0.8}\NormalTok{)}
\NormalTok{labels <-}\StringTok{ }\KeywordTok{c}\NormalTok{(}\StringTok{"setosa"}\NormalTok{, }\StringTok{"versicolor"}\NormalTok{, }\StringTok{"virginica"}\NormalTok{)}
\NormalTok{p }\OperatorTok{+}\StringTok{ }\KeywordTok{geom_density}\NormalTok{(}\DataTypeTok{alpha =} \FloatTok{0.5}\NormalTok{, }\DataTypeTok{show.legend =} \OtherTok{FALSE}\NormalTok{) }\OperatorTok{+}
\StringTok{  }\KeywordTok{annotate}\NormalTok{(}\StringTok{"text"}\NormalTok{, }\DataTypeTok{x =}\NormalTok{ x_pos, }\DataTypeTok{y =}\NormalTok{ y_pos, }\DataTypeTok{label =}\NormalTok{ labels,}
           \DataTypeTok{family =} \StringTok{"Helvetica"}\NormalTok{)}
\end{Highlighting}
\end{Shaded}

\includegraphics[height=5cm]{SappoRoR9_files/figure-beamer/annotate-1}

\end{frame}

\begin{frame}[fragile]{数式}

\texttt{parse\ =\ TRUE}で数式を解釈して表示

\begin{Shaded}
\begin{Highlighting}[]
\KeywordTok{ggplot}\NormalTok{(}\KeywordTok{data.frame}\NormalTok{(}\DataTypeTok{x =} \KeywordTok{seq}\NormalTok{(}\OperatorTok{-}\DecValTok{5}\NormalTok{, }\DecValTok{5}\NormalTok{, }\FloatTok{0.01}\NormalTok{)), }\KeywordTok{aes}\NormalTok{(x)) }\OperatorTok{+}
\StringTok{  }\KeywordTok{stat_function}\NormalTok{(}\DataTypeTok{fun =} \ControlFlowTok{function}\NormalTok{(x) \{}\DecValTok{1} \OperatorTok{/}\StringTok{ }\NormalTok{(}\DecValTok{1} \OperatorTok{+}\StringTok{ }\KeywordTok{exp}\NormalTok{(}\OperatorTok{-}\NormalTok{x))\}) }\OperatorTok{+}
\StringTok{  }\KeywordTok{annotate}\NormalTok{(}\StringTok{"text"}\NormalTok{, }\DataTypeTok{x =} \OperatorTok{-}\DecValTok{5}\NormalTok{, }\DataTypeTok{y =} \FloatTok{0.875}\NormalTok{,}
      \DataTypeTok{label =} \KeywordTok{paste}\NormalTok{(}\StringTok{"italic(y) == frac(1,"}\NormalTok{, }
                    \StringTok{"1 + exp(-(beta[0] + beta[1]*italic(x))))"}\NormalTok{),}
      \DataTypeTok{parse =} \OtherTok{TRUE}\NormalTok{, }\DataTypeTok{size =} \DecValTok{4}\NormalTok{, }\DataTypeTok{hjust =} \DecValTok{0}\NormalTok{)}
\end{Highlighting}
\end{Shaded}

\includegraphics[height=4.5cm]{SappoRoR9_files/figure-beamer/expression-1}

\end{frame}

\begin{frame}[fragile]{棒グラフ}

\textbf{\texttt{geom\_bar}}関数

\begin{Shaded}
\begin{Highlighting}[]
\KeywordTok{set.seed}\NormalTok{(}\DecValTok{1}\NormalTok{); x <-}\StringTok{ }\KeywordTok{rpois}\NormalTok{(}\DecValTok{200}\NormalTok{, }\DecValTok{2}\NormalTok{)}
\NormalTok{p <-}\StringTok{ }\KeywordTok{ggplot}\NormalTok{(}\KeywordTok{data.frame}\NormalTok{(}\DataTypeTok{x =}\NormalTok{ x), }\KeywordTok{aes}\NormalTok{(x))}
\NormalTok{p }\OperatorTok{+}\StringTok{ }\KeywordTok{geom_bar}\NormalTok{(}\DataTypeTok{fill =} \StringTok{"grey50"}\NormalTok{)}
\end{Highlighting}
\end{Shaded}

\includegraphics[height=5.25cm]{SappoRoR9_files/figure-beamer/geom_bar-1}

\end{frame}

\begin{frame}[fragile]{stat\_*関数}

統計的変換, 例) \textbf{\texttt{stat\_count}}:
データの頻度分布を計算する。

\begin{Shaded}
\begin{Highlighting}[]
\NormalTok{p }\OperatorTok{+}\StringTok{ }\KeywordTok{stat_count}\NormalTok{(}\DataTypeTok{geom =} \StringTok{"bar"}\NormalTok{, }\DataTypeTok{fill =} \StringTok{"grey50"}\NormalTok{)}
\end{Highlighting}
\end{Shaded}

\includegraphics[height=4cm]{SappoRoR9_files/figure-beamer/stat_count-1}

\begin{itemize}
\tightlist
\item
  たいていの\texttt{geom}には対応する\texttt{stat}関数がある。
\item
  \texttt{geom}ではなくて\texttt{stat}で描画レイヤーを追加することも可能。
\end{itemize}

\end{frame}

\begin{frame}[fragile]{3次元のデータ}

\begin{Shaded}
\begin{Highlighting}[]
\NormalTok{r <-}\StringTok{ }\ControlFlowTok{function}\NormalTok{(x, y) (}\OperatorTok{-}\FloatTok{0.00002} \OperatorTok{*}\StringTok{ }\NormalTok{(x }\OperatorTok{-}\StringTok{ }\DecValTok{40}\NormalTok{)}\OperatorTok{^}\DecValTok{2} \OperatorTok{+}\StringTok{ }\FloatTok{0.05}\NormalTok{) }\OperatorTok{*}\StringTok{ }\FloatTok{0.04} \OperatorTok{*}\StringTok{ }\NormalTok{y}
\NormalTok{raster_data <-}\StringTok{ }\KeywordTok{expand.grid}\NormalTok{(}\DataTypeTok{x =} \DecValTok{0}\OperatorTok{:}\DecValTok{100}\NormalTok{, }\DataTypeTok{y =} \DecValTok{0}\OperatorTok{:}\DecValTok{100}\NormalTok{) }\OperatorTok
\StringTok{  }\KeywordTok{mutate}\NormalTok{(}\DataTypeTok{z =} \KeywordTok{r}\NormalTok{(x, y))}
\KeywordTok{print}\NormalTok{(raster_data[}\DecValTok{5000}\OperatorTok{:}\DecValTok{5005}\NormalTok{, ])}
\end{Highlighting}
\end{Shaded}

\begin{verbatim}
##       x  y         z
## 5000 50 49 0.0940800
## 5001 51 49 0.0932568
## 5002 52 49 0.0923552
## 5003 53 49 0.0913752
## 5004 54 49 0.0903168
## 5005 55 49 0.0891800
\end{verbatim}

\footnotesize データを生成する関数は,島谷(2017) 5.3節より。

\end{frame}

\begin{frame}[fragile]{ラスターで表示}

\textbf{\texttt{geom\_raster}}関数

\begin{Shaded}
\begin{Highlighting}[]
\NormalTok{p <-}\StringTok{ }\KeywordTok{ggplot}\NormalTok{(raster_data, }\KeywordTok{aes}\NormalTok{(x, y))}
\NormalTok{p }\OperatorTok{+}\StringTok{ }\KeywordTok{geom_raster}\NormalTok{(}\KeywordTok{aes}\NormalTok{(}\DataTypeTok{fill =}\NormalTok{ z)) }\OperatorTok{+}\StringTok{ }\KeywordTok{coord_fixed}\NormalTok{()}
\end{Highlighting}
\end{Shaded}

\includegraphics[height=5.5cm]{SappoRoR9_files/figure-beamer/geom_raster-1}

\end{frame}

\begin{frame}[fragile]{等高線で表示}

\textbf{\texttt{geom\_contour}}関数

\begin{Shaded}
\begin{Highlighting}[]
\NormalTok{p }\OperatorTok{+}\StringTok{ }\KeywordTok{geom_contour}\NormalTok{(}\KeywordTok{aes}\NormalTok{(}\DataTypeTok{z =}\NormalTok{ z), }\DataTypeTok{binwidth =} \FloatTok{0.01}\NormalTok{) }\OperatorTok{+}\StringTok{ }\KeywordTok{coord_fixed}\NormalTok{()}
\end{Highlighting}
\end{Shaded}

\includegraphics[height=5.5cm]{SappoRoR9_files/figure-beamer/geom_contour-1}

\end{frame}

\begin{frame}[fragile]{地図データ}

国土交通省国土数値情報行政区域データ(N03-170101\_01)を,\textbf{\texttt{kokudosuuchi}}パッケージを利用して
よみこむ。

\begin{Shaded}
\begin{Highlighting}[]
\KeywordTok{library}\NormalTok{(kokudosuuchi)}

\NormalTok{hokkaido_url <-}\StringTok{ }\KeywordTok{getKSJURL}\NormalTok{(}\StringTok{"N03"}\NormalTok{, }\DataTypeTok{prefCode =} \DecValTok{1}\NormalTok{) }\OperatorTok
\StringTok{  }\KeywordTok{filter}\NormalTok{(year }\OperatorTok{==}\StringTok{ }\DecValTok{2017}\NormalTok{)}
\NormalTok{hokkaido_}\DecValTok{2017}\NormalTok{ <-}\StringTok{ }\KeywordTok{getKSJData}\NormalTok{(hokkaido_url}\OperatorTok{$}\NormalTok{zipFileUrl)}
\end{Highlighting}
\end{Shaded}

\end{frame}

\begin{frame}[fragile]{地図データの整形}

\begin{Shaded}
\begin{Highlighting}[]
\KeywordTok{library}\NormalTok{(sf)}

\NormalTok{sapporo <-}\StringTok{ }\NormalTok{hokkaido_}\DecValTok{2017}\NormalTok{[[}\DecValTok{1}\NormalTok{]] }\OperatorTok
\StringTok{  }\KeywordTok{filter}\NormalTok{(N03_}\DecValTok{003} \OperatorTok{==}\StringTok{ "札幌市"}\NormalTok{)}
\NormalTok{sapporo_df <-}\StringTok{ }\KeywordTok{as_Spatial}\NormalTok{(sapporo}\OperatorTok{$}\NormalTok{geometry) }\OperatorTok
\StringTok{  }\KeywordTok{fortify}\NormalTok{() }\OperatorTok
\StringTok{  }\KeywordTok{mutate}\NormalTok{(}\DataTypeTok{id =} \KeywordTok{recode_factor}\NormalTok{(id,}
                            \DataTypeTok{ID1 =} \StringTok{"中央区"}\NormalTok{,}
                            \DataTypeTok{ID2 =} \StringTok{"北区"}\NormalTok{,}
                            \DataTypeTok{ID3 =} \StringTok{"東区"}\NormalTok{,}
                            \DataTypeTok{ID4 =} \StringTok{"白石区"}\NormalTok{,}
                            \DataTypeTok{ID5 =} \StringTok{"豊平区"}\NormalTok{,}
                            \DataTypeTok{ID6 =} \StringTok{"南区"}\NormalTok{,}
                            \DataTypeTok{ID7 =} \StringTok{"西区"}\NormalTok{,}
                            \DataTypeTok{ID8 =} \StringTok{"厚別区"}\NormalTok{,}
                            \DataTypeTok{ID9 =} \StringTok{"手稲区"}\NormalTok{,}
                            \DataTypeTok{ID10 =} \StringTok{"清田区"}\NormalTok{,}
                            \DataTypeTok{.default =} \StringTok{""}\NormalTok{))}
\end{Highlighting}
\end{Shaded}

\end{frame}

\begin{frame}[fragile]{地図データの描画}

地図のオブジェクトを用意して\textbf{\texttt{geom\_polygon}}関数で描画。

\begin{Shaded}
\begin{Highlighting}[]
\KeywordTok{ggplot}\NormalTok{(sapporo_df) }\OperatorTok{+}\StringTok{ }\KeywordTok{coord_fixed}\NormalTok{() }\OperatorTok{+}
\StringTok{  }\KeywordTok{geom_polygon}\NormalTok{(}\KeywordTok{aes}\NormalTok{(long, lat, }\DataTypeTok{fill =}\NormalTok{ id), }\DataTypeTok{colour =} \StringTok{"grey"}\NormalTok{) }\OperatorTok{+}
\StringTok{  }\KeywordTok{theme_bw}\NormalTok{(}\DataTypeTok{base_family =} \StringTok{"IPAexGothic"}\NormalTok{)}
\end{Highlighting}
\end{Shaded}

\includegraphics[height=3.6cm]{SappoRoR9_files/figure-beamer/polygon-1}

\small ggplot2開発版には\textbf{\texttt{geom\_sf}}関数がある。
いずれは,地図描画にはそちらがつかわれるようになるはず。

\end{frame}

\section{おわりに}

\begin{frame}{参考リンク}

きょう紹介できたのはごく一部。下のリンクなどを参考に。

\begin{itemize}
\tightlist
\item
  ggplot2(公式ウェブサイト)

  \begin{itemize}
  \tightlist
  \item
    \url{http://ggplot2.tidyverse.org/}
  \end{itemize}
\item
  ggplot2に関する資料(前田和寛 `@kazutan' さん)

  \begin{itemize}
  \tightlist
  \item
    \url{https://kazutan.github.io/kazutanR/ggplot2_links.html}
  \end{itemize}
\item
  ggplot2 --- きれいなグラフを簡単に合理的に(岩嵜航さん)

  \begin{itemize}
  \tightlist
  \item
    \url{https://heavywatal.github.io/rstats/ggplot2.html}
  \end{itemize}
\item
  グラフ描画ggplot2の辞書的まとめ20のコード(MrUnadonさん)

  \begin{itemize}
  \tightlist
  \item
    \url{https://mrunadon.github.io/ggplot2/}
  \end{itemize}
\end{itemize}

\end{frame}

\end{document}
