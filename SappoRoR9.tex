\documentclass[ignorenonframetext,]{beamer}
\setbeamertemplate{caption}[numbered]
\setbeamertemplate{caption label separator}{: }
\setbeamercolor{caption name}{fg=normal text.fg}
\beamertemplatenavigationsymbolsempty
\usepackage{lmodern}
\usepackage{amssymb,amsmath}
\usepackage{ifxetex,ifluatex}
\usepackage{fixltx2e} % provides \textsubscript
\ifnum 0\ifxetex 1\fi\ifluatex 1\fi=0 % if pdftex
  \usepackage[T1]{fontenc}
  \usepackage[utf8]{inputenc}
\else % if luatex or xelatex
  \ifxetex
    \usepackage{mathspec}
  \else
    \usepackage{fontspec}
  \fi
  \defaultfontfeatures{Ligatures=TeX,Scale=MatchLowercase}
\fi
% use upquote if available, for straight quotes in verbatim environments
\IfFileExists{upquote.sty}{\usepackage{upquote}}{}
% use microtype if available
\IfFileExists{microtype.sty}{%
\usepackage{microtype}
\UseMicrotypeSet[protrusion]{basicmath} % disable protrusion for tt fonts
}{}
\newif\ifbibliography
\hypersetup{
            pdftitle={ggplot2をつかってみよう},
            pdfauthor={伊東宏樹},
            pdfborder={0 0 0},
            breaklinks=true}
\urlstyle{same}  % don't use monospace font for urls
\usepackage{color}
\usepackage{fancyvrb}
\newcommand{\VerbBar}{|}
\newcommand{\VERB}{\Verb[commandchars=\\\{\}]}
\DefineVerbatimEnvironment{Highlighting}{Verbatim}{commandchars=\\\{\}}
% Add ',fontsize=\small' for more characters per line
\usepackage{framed}
\definecolor{shadecolor}{RGB}{248,248,248}
\newenvironment{Shaded}{\begin{snugshade}}{\end{snugshade}}
\newcommand{\KeywordTok}[1]{\textcolor[rgb]{0.13,0.29,0.53}{\textbf{#1}}}
\newcommand{\DataTypeTok}[1]{\textcolor[rgb]{0.13,0.29,0.53}{#1}}
\newcommand{\DecValTok}[1]{\textcolor[rgb]{0.00,0.00,0.81}{#1}}
\newcommand{\BaseNTok}[1]{\textcolor[rgb]{0.00,0.00,0.81}{#1}}
\newcommand{\FloatTok}[1]{\textcolor[rgb]{0.00,0.00,0.81}{#1}}
\newcommand{\ConstantTok}[1]{\textcolor[rgb]{0.00,0.00,0.00}{#1}}
\newcommand{\CharTok}[1]{\textcolor[rgb]{0.31,0.60,0.02}{#1}}
\newcommand{\SpecialCharTok}[1]{\textcolor[rgb]{0.00,0.00,0.00}{#1}}
\newcommand{\StringTok}[1]{\textcolor[rgb]{0.31,0.60,0.02}{#1}}
\newcommand{\VerbatimStringTok}[1]{\textcolor[rgb]{0.31,0.60,0.02}{#1}}
\newcommand{\SpecialStringTok}[1]{\textcolor[rgb]{0.31,0.60,0.02}{#1}}
\newcommand{\ImportTok}[1]{#1}
\newcommand{\CommentTok}[1]{\textcolor[rgb]{0.56,0.35,0.01}{\textit{#1}}}
\newcommand{\DocumentationTok}[1]{\textcolor[rgb]{0.56,0.35,0.01}{\textbf{\textit{#1}}}}
\newcommand{\AnnotationTok}[1]{\textcolor[rgb]{0.56,0.35,0.01}{\textbf{\textit{#1}}}}
\newcommand{\CommentVarTok}[1]{\textcolor[rgb]{0.56,0.35,0.01}{\textbf{\textit{#1}}}}
\newcommand{\OtherTok}[1]{\textcolor[rgb]{0.56,0.35,0.01}{#1}}
\newcommand{\FunctionTok}[1]{\textcolor[rgb]{0.00,0.00,0.00}{#1}}
\newcommand{\VariableTok}[1]{\textcolor[rgb]{0.00,0.00,0.00}{#1}}
\newcommand{\ControlFlowTok}[1]{\textcolor[rgb]{0.13,0.29,0.53}{\textbf{#1}}}
\newcommand{\OperatorTok}[1]{\textcolor[rgb]{0.81,0.36,0.00}{\textbf{#1}}}
\newcommand{\BuiltInTok}[1]{#1}
\newcommand{\ExtensionTok}[1]{#1}
\newcommand{\PreprocessorTok}[1]{\textcolor[rgb]{0.56,0.35,0.01}{\textit{#1}}}
\newcommand{\AttributeTok}[1]{\textcolor[rgb]{0.77,0.63,0.00}{#1}}
\newcommand{\RegionMarkerTok}[1]{#1}
\newcommand{\InformationTok}[1]{\textcolor[rgb]{0.56,0.35,0.01}{\textbf{\textit{#1}}}}
\newcommand{\WarningTok}[1]{\textcolor[rgb]{0.56,0.35,0.01}{\textbf{\textit{#1}}}}
\newcommand{\AlertTok}[1]{\textcolor[rgb]{0.94,0.16,0.16}{#1}}
\newcommand{\ErrorTok}[1]{\textcolor[rgb]{0.64,0.00,0.00}{\textbf{#1}}}
\newcommand{\NormalTok}[1]{#1}

% Prevent slide breaks in the middle of a paragraph:
\widowpenalties 1 10000
\raggedbottom

\AtBeginPart{
  \let\insertpartnumber\relax
  \let\partname\relax
  \frame{\partpage}
}
\AtBeginSection{
  \ifbibliography
  \else
    \let\insertsectionnumber\relax
    \let\sectionname\relax
    \frame{\sectionpage}
  \fi
}
\AtBeginSubsection{
  \let\insertsubsectionnumber\relax
  \let\subsectionname\relax
  \frame{\subsectionpage}
}

\setlength{\parindent}{0pt}
\setlength{\parskip}{6pt plus 2pt minus 1pt}
\setlength{\emergencystretch}{3em}  % prevent overfull lines
\providecommand{\tightlist}{%
  \setlength{\itemsep}{0pt}\setlength{\parskip}{0pt}}
\setcounter{secnumdepth}{0}
%% PDFメタデータの文字化け防止
% https://blog.miz-ar.info/2015/09/latex-hyperref-tips/
% https://tex.stackexchange.com/questions/24445/hyperref-lualatex-and-unicode-bookmarks-issue-garbled-page-numbers-in-ar-for-l
\hypersetup{%
  pdfencoding=auto
}

%% Fonts
\usepackage[T1]{fontenc}
\usepackage{textcomp}
\usepackage[scale=1.0]{tgheros}
\usepackage[scaled]{beramono}

%% Japanese font
\usepackage{luatexja-otf}
\usepackage[match,deluxe,expert,noto-otf]{luatexja-preset}
\renewcommand{\kanjifamilydefault}{\gtdefault}

%% change fontsize of R code
% https://stackoverflow.com/questions/38323331/code-chunk-font-size-in-beamer-with-knitr-and-latex
\let\oldShaded\Shaded
\let\endoldShaded\endShaded
\renewenvironment{Shaded}{\footnotesize\oldShaded}{\endoldShaded}

%% change fontsize of output
\let\oldverbatim\verbatim
\let\endoldverbatim\endverbatim
\renewenvironment{verbatim}{\footnotesize\oldverbatim}{\endoldverbatim}

%% Title font size
\setbeamerfont{title}{size=\Huge, series=\bfseries}
\setbeamerfont{section title}{size=\LARGE, series=\bfseries}
\setbeamerfont{frametitle}{size=\Large, series=\bfseries}

\title{ggplot2をつかってみよう}
\author{伊東宏樹}
\date{2019/3/2}

\begin{document}
\frame{\titlepage}

\section{はじめに}

\begin{frame}{今回の発表に使用したコード・データ}

\url{https://github.com/ito4303/SappoRoR9} で公開

\end{frame}

\begin{frame}{このごろよくみるこのようなグラフ}

\includegraphics[height=7cm]{SappoRoR9_files/figure-beamer/introduction-1}

\end{frame}

\begin{frame}{ggplot2}

\LARGE

\begin{itemize}[<+->]
\tightlist
\item
  ggplot2パッケージ
\end{itemize}

\normalsize

\begin{itemize}[<+->]
\tightlist
\item
  gg: Grammer of Graphics
\end{itemize}

\large

\begin{itemize}[<+->]
\tightlist
\item
  統一された文法でさまざまな種類のグラフをえがける。
\item
  関連パッケージもどんどんできている。
\item
  GGally, ggmcmc, ggmap, ggthemes, gghighlight, egg\dots
\end{itemize}

\end{frame}

\begin{frame}{開発者など}

\begin{itemize}
\tightlist
\item
  開発者: Hadley Wickham \& Winston Chang
\item
  ウェブサイト: \url{http://ggplot2.tidyverse.org/}
\item
  開発履歴 \footnote{https://github.com/tidyverse/ggplot2/releases}

  \begin{itemize}
  \tightlist
  \item
    2015-01-10 ver. 1.0.0
  \item
    2015-12-19 ver. 2.0.0
  \item
    2016-03-02 ver. 2.1.0
  \item
    2016-11-15 ver. 2.2.0
  \item
    2016-12-31 ver. 2.2.1
  \item
    2018-07-04 ver. 3.0.0
  \item
    2018-10-25 ver. 3.1.0
  \end{itemize}
\end{itemize}

\end{frame}

\begin{frame}[fragile]{ggplot関数}

\texttt{ggplot}オブジェクトを生成する。

\begin{Shaded}
\begin{Highlighting}[]
\NormalTok{p <-}\StringTok{ }\KeywordTok{ggplot}\NormalTok{(}\DataTypeTok{data =}\NormalTok{ iris)}
\KeywordTok{print}\NormalTok{(p)}
\end{Highlighting}
\end{Shaded}

\includegraphics[height=6cm]{SappoRoR9_files/figure-beamer/ggplot_func-1}

\end{frame}

\begin{frame}[fragile]{geom\_*関数}

\begin{itemize}
\tightlist
\item
  \textbf{\texttt{geom\_point}} 関数:
  散布図を描画(レイヤーに追加)する関数

  \begin{itemize}
  \tightlist
  \item
    \texttt{mapping} 引数: 変数のマッピングを渡す。
  \end{itemize}
\item
  \textbf{\texttt{aes}} 関数: 変数とグラフ要素とのaesthetic
  mappingを生成
\end{itemize}

\textcolor{red}{\bfseries \texttt{ggplot}オブジェクトに`+'演算子で,レイヤーを追加する。}

\begin{Shaded}
\begin{Highlighting}[]
\NormalTok{p }\OperatorTok{+}\StringTok{ }\KeywordTok{geom_point}\NormalTok{(}\DataTypeTok{mapping =} \KeywordTok{aes}\NormalTok{(}\DataTypeTok{x =}\NormalTok{ Sepal.Length, }\DataTypeTok{y =}\NormalTok{ Sepal.Width))}
\end{Highlighting}
\end{Shaded}

\includegraphics[height=5cm]{SappoRoR9_files/figure-beamer/geom_func-1}

\end{frame}

\begin{frame}[fragile]{種類ごとに色をかえる}

\textbf{\texttt{aes}}関数の\texttt{colour}(\texttt{color}でもよい)引数を指定する。

\begin{Shaded}
\begin{Highlighting}[]
\NormalTok{p }\OperatorTok{+}\StringTok{ }\KeywordTok{geom_point}\NormalTok{(}\DataTypeTok{mapping =} \KeywordTok{aes}\NormalTok{(}\DataTypeTok{x =}\NormalTok{ Sepal.Length, }\DataTypeTok{y =}\NormalTok{ Sepal.Width,}
                             \DataTypeTok{colour =}\NormalTok{ Species))}
\end{Highlighting}
\end{Shaded}

\includegraphics[height=5.5cm]{SappoRoR9_files/figure-beamer/colour-1}

\end{frame}

\begin{frame}[fragile]{こうしてもおなじ}

\begin{Shaded}
\begin{Highlighting}[]
\KeywordTok{ggplot}\NormalTok{(}\DataTypeTok{data =}\NormalTok{ iris,}
       \DataTypeTok{mapping =} \KeywordTok{aes}\NormalTok{(}\DataTypeTok{x =}\NormalTok{ Sepal.Length, }\DataTypeTok{y =}\NormalTok{ Sepal.Width,}
                     \DataTypeTok{colour =}\NormalTok{ Species)) }\OperatorTok{+}
\StringTok{  }\KeywordTok{geom_point}\NormalTok{()}
\end{Highlighting}
\end{Shaded}

\includegraphics[height=5.5cm]{SappoRoR9_files/figure-beamer/colour2-1}

\end{frame}

\begin{frame}[fragile]{すべての点の色をかえる}

\textbf{\texttt{aes}}関数の外で\texttt{colour}を指定する(\textbf{\texttt{geom\_point}}関数の\texttt{colour}引数に指定する)と,すべての点の色が指定した色になる。

\begin{Shaded}
\begin{Highlighting}[]
\NormalTok{p }\OperatorTok{+}\StringTok{ }\KeywordTok{geom_point}\NormalTok{(}\DataTypeTok{mapping =} \KeywordTok{aes}\NormalTok{(}\DataTypeTok{x =}\NormalTok{ Sepal.Length, }\DataTypeTok{y =}\NormalTok{ Sepal.Width),}
               \DataTypeTok{colour =} \StringTok{"red"}\NormalTok{)}
\end{Highlighting}
\end{Shaded}

\includegraphics[height=5.5cm]{SappoRoR9_files/figure-beamer/colour3-1}

\end{frame}

\begin{frame}[fragile]{種類ごとに点の形をかえる}

\textbf{\texttt{aes}}関数の\texttt{shape}引数を指定する。

\begin{Shaded}
\begin{Highlighting}[]
\NormalTok{p }\OperatorTok{+}\StringTok{ }\KeywordTok{geom_point}\NormalTok{(}\KeywordTok{aes}\NormalTok{(}\DataTypeTok{x =}\NormalTok{ Sepal.Length, }\DataTypeTok{y =}\NormalTok{ Sepal.Width,}
                   \DataTypeTok{colour =}\NormalTok{ Species, }\DataTypeTok{shape =}\NormalTok{ Species))}
\end{Highlighting}
\end{Shaded}

\includegraphics[height=5cm]{SappoRoR9_files/figure-beamer/point_shape-1}

\end{frame}

\begin{frame}[fragile]{点を大きくする}

\textbf{\texttt{geom\_point}}関数の\texttt{size}引数を指定する。

\begin{Shaded}
\begin{Highlighting}[]
\NormalTok{p2 <-}\StringTok{ }\NormalTok{p }\OperatorTok{+}\StringTok{ }\KeywordTok{geom_point}\NormalTok{(}\KeywordTok{aes}\NormalTok{(}\DataTypeTok{x =}\NormalTok{ Sepal.Length, }\DataTypeTok{y =}\NormalTok{ Sepal.Width,}
                         \DataTypeTok{colour =}\NormalTok{ Species, }\DataTypeTok{shape =}\NormalTok{ Species),}
                     \DataTypeTok{size =} \DecValTok{4}\NormalTok{, }\DataTypeTok{alpha =} \FloatTok{0.7}\NormalTok{)}
\KeywordTok{print}\NormalTok{(p2)}
\end{Highlighting}
\end{Shaded}

\includegraphics[height=5cm]{SappoRoR9_files/figure-beamer/point_size-1}

\end{frame}

\begin{frame}[fragile]{色を指定する}

\textbf{\texttt{scale\_colour\_manual}}関数で,任意の色を指定できる。

\begin{Shaded}
\begin{Highlighting}[]
\NormalTok{p2 }\OperatorTok{+}\StringTok{ }\KeywordTok{scale_colour_manual}\NormalTok{(}\DataTypeTok{values =} \KeywordTok{c}\NormalTok{(}\StringTok{"black"}\NormalTok{, }\StringTok{"red"}\NormalTok{, }\StringTok{"#3355FF"}\NormalTok{))}
\end{Highlighting}
\end{Shaded}

\includegraphics[height=5cm]{SappoRoR9_files/figure-beamer/spec_colour-1}

\end{frame}

\begin{frame}[fragile]{漢字をつかう}

ここではIPAexゴシックフォントを使用した。

\begin{Shaded}
\begin{Highlighting}[]
\NormalTok{spp <-}\StringTok{ }\KeywordTok{c}\NormalTok{(}\StringTok{"ヒオウギアヤメ"}\NormalTok{, }\StringTok{"イリス・ヴェルシカラー"}\NormalTok{,}
         \StringTok{"イリス・ヴァージニカ"}\NormalTok{)}
\NormalTok{p3 <-}\StringTok{ }\NormalTok{p2 }\OperatorTok{+}\StringTok{ }\KeywordTok{labs}\NormalTok{(}\DataTypeTok{x =} \StringTok{"萼長 (cm)"}\NormalTok{, }\DataTypeTok{y =} \StringTok{"萼幅(cm)"}\NormalTok{) }\OperatorTok{+}
\StringTok{  }\KeywordTok{scale_colour_discrete}\NormalTok{(}\DataTypeTok{name =} \StringTok{"種"}\NormalTok{, }\DataTypeTok{labels =}\NormalTok{ spp) }\OperatorTok{+}
\StringTok{  }\KeywordTok{scale_shape_discrete}\NormalTok{(}\DataTypeTok{name =} \StringTok{"種"}\NormalTok{, }\DataTypeTok{labels =}\NormalTok{ spp) }\OperatorTok{+}
\StringTok{  }\KeywordTok{scale_size_continuous}\NormalTok{(}\DataTypeTok{name =} \StringTok{"萼長 (cm)"}\NormalTok{) }\OperatorTok{+}
\StringTok{  }\KeywordTok{theme}\NormalTok{(}\DataTypeTok{text =} \KeywordTok{element_text}\NormalTok{(}\DataTypeTok{family =} \StringTok{"IPAexGothic"}\NormalTok{, }\DataTypeTok{size =} \DecValTok{10}\NormalTok{))}
\end{Highlighting}
\end{Shaded}

文字コードはUTF-8にしておくとよい。\footnote{日本語版WindowsのR.exeでは文字化けするかも。}

Windowsでは\textbf{\texttt{windowsFonts}}関数でフォントの設定をしておくか,\textbf{extrafont}パッケージをインストールして,必要な設定をしておく。

\end{frame}

\begin{frame}{漢字をつかった表示}

\includegraphics[height=7cm]{SappoRoR9_files/figure-beamer/Japanese_font2-1}

\end{frame}

\begin{frame}[fragile]{図をファイルに保存する}

\textbf{\texttt{ggsave}}関数などをつかう。
\footnote{R標準のグラフィックデバイス関数による出力もふつうにできる。}

\begin{Shaded}
\begin{Highlighting}[]
\KeywordTok{ggsave}\NormalTok{(}\StringTok{"iris.pdf"}\NormalTok{, }\DataTypeTok{device =}\NormalTok{ cairo_pdf,}
       \DataTypeTok{width =} \DecValTok{12}\NormalTok{, }\DataTypeTok{height =} \DecValTok{8}\NormalTok{, }\DataTypeTok{units =} \StringTok{"cm"}\NormalTok{)}
\end{Highlighting}
\end{Shaded}

日本語フォントを埋め込んだPDFを出力するには\\
\texttt{device\ =\ cairo\_pdf}\\
とする。\footnote{macOSではXQuartzのインストールが必要になるかもしれない。}
ただし,すべてのフォントでうまくいくとは限らない。

WindowsでPDFに日本語フォントを埋め込むためには,\textbf{extrafont}パッケージを利用する必要がある。
\footnote{ただし,OpenTypeフォントは利用不可らしい。}

\vspace{12pt}

macOSでは,\textbf{\texttt{ggsave}}関数のほか,\textbf{\texttt{quartz.save}}関数も利用可能。

\end{frame}

\begin{frame}[fragile]{themeをかえる(1)}

\begin{Shaded}
\begin{Highlighting}[]
\NormalTok{p3 }\OperatorTok{+}\StringTok{ }\KeywordTok{theme_bw}\NormalTok{(}\DataTypeTok{base_family =} \StringTok{"IPAexGothic"}\NormalTok{)}
\end{Highlighting}
\end{Shaded}

\includegraphics[height=6cm]{SappoRoR9_files/figure-beamer/theme_bw-1}

\end{frame}

\begin{frame}[fragile]{themeをかえる(2)}

\begin{Shaded}
\begin{Highlighting}[]
\NormalTok{p3 }\OperatorTok{+}\StringTok{ }\KeywordTok{theme_classic}\NormalTok{(}\DataTypeTok{base_family =} \StringTok{"IPAexGothic"}\NormalTok{)}
\end{Highlighting}
\end{Shaded}

\includegraphics[height=6cm]{SappoRoR9_files/figure-beamer/theme_classic-1}

\end{frame}

\begin{frame}[fragile]{themeをかえる(3)}

\begin{Shaded}
\begin{Highlighting}[]
\KeywordTok{library}\NormalTok{(ggthemes)}
\NormalTok{p2 }\OperatorTok{+}\StringTok{ }\KeywordTok{theme_excel}\NormalTok{()}
\end{Highlighting}
\end{Shaded}

\includegraphics[height=6cm]{SappoRoR9_files/figure-beamer/theme_excel-1}

\end{frame}

\begin{frame}[fragile]{facet: 分割して表示}

\textbf{\texttt{facet\_wrap}}関数を使用して,種ごとに分割して表示。

\begin{Shaded}
\begin{Highlighting}[]
\NormalTok{p3 }\OperatorTok{+}\StringTok{ }\KeywordTok{facet_wrap}\NormalTok{(}\OperatorTok{~}\StringTok{ }\NormalTok{Species) }\OperatorTok{+}\StringTok{ }\KeywordTok{theme}\NormalTok{(}\DataTypeTok{legend.position =} \StringTok{"none"}\NormalTok{)}
\end{Highlighting}
\end{Shaded}

\includegraphics[height=5.5cm]{SappoRoR9_files/figure-beamer/facet-1}

\end{frame}

\section{ほかの種類のグラフは}

\begin{frame}[fragile]{\textbf{\texttt{geom\_*}} 関数}

\begin{itemize}
\tightlist
\item
  \textbf{\texttt{geom\_bar}}
\item
  \textbf{\texttt{geom\_boxplot}}
\item
  \textbf{\texttt{geom\_contour}}
\item
  \textbf{\texttt{geom\_density}}
\item
  \textbf{\texttt{geom\_errorbar}}
\item
  \textbf{\texttt{geom\_histogram}}
\item
  \textbf{\texttt{geom\_line}}
\item
  \textbf{\texttt{geom\_polygon}}
\item
  \textbf{\texttt{geom\_ribbon}}
\item
  \textbf{\texttt{geom\_smooth}}
\item
  \textbf{\texttt{geom\_tile}}
\end{itemize}

などなど

\end{frame}

\begin{frame}[fragile]{折れ線グラフ}

\textbf{\texttt{geom\_line}}関数

ナイル川のデータ

\begin{Shaded}
\begin{Highlighting}[]
\NormalTok{(p <-}\StringTok{ }\KeywordTok{ggplot}\NormalTok{(df_Nile, }\KeywordTok{aes}\NormalTok{(}\DataTypeTok{x =}\NormalTok{ Year, }\DataTypeTok{y =}\NormalTok{ Level)) }\OperatorTok{+}\StringTok{ }\KeywordTok{geom_line}\NormalTok{())}
\end{Highlighting}
\end{Shaded}

\includegraphics[height=5.5cm]{SappoRoR9_files/figure-beamer/geom_line1-1}

\end{frame}

\begin{frame}[fragile]{折れ線グラフ+点}

\textbf{\texttt{geom\_line}}と\textbf{\texttt{geom\_point}}の両方を使用。

\begin{Shaded}
\begin{Highlighting}[]
\NormalTok{p }\OperatorTok{+}\StringTok{ }\KeywordTok{geom_line}\NormalTok{(}\DataTypeTok{linetype =} \DecValTok{2}\NormalTok{) }\OperatorTok{+}\StringTok{ }\KeywordTok{geom_point}\NormalTok{(}\DataTypeTok{size =} \DecValTok{2}\NormalTok{)}
\end{Highlighting}
\end{Shaded}

\includegraphics[height=5.5cm]{SappoRoR9_files/figure-beamer/geom_line2-1}

\end{frame}

\begin{frame}[fragile]{平滑化曲線}

\textbf{\texttt{geom\_smooth}}を使用。

\begin{Shaded}
\begin{Highlighting}[]
\NormalTok{p }\OperatorTok{+}\StringTok{ }\KeywordTok{geom_line}\NormalTok{(}\DataTypeTok{linetype =} \DecValTok{2}\NormalTok{) }\OperatorTok{+}\StringTok{ }\KeywordTok{geom_smooth}\NormalTok{(}\DataTypeTok{method =} \StringTok{"loess"}\NormalTok{)}
\end{Highlighting}
\end{Shaded}

\includegraphics[height=5.5cm]{SappoRoR9_files/figure-beamer/geom_line3-1}

\end{frame}

\begin{frame}[fragile]{棒グラフ}

\textbf{\texttt{geom\_bar}}関数

\begin{Shaded}
\begin{Highlighting}[]
\KeywordTok{set.seed}\NormalTok{(}\DecValTok{1}\NormalTok{); x <-}\StringTok{ }\KeywordTok{rpois}\NormalTok{(}\DecValTok{200}\NormalTok{, }\DecValTok{2}\NormalTok{)}
\NormalTok{p <-}\StringTok{ }\KeywordTok{ggplot}\NormalTok{(}\KeywordTok{data.frame}\NormalTok{(}\DataTypeTok{x =}\NormalTok{ x), }\KeywordTok{aes}\NormalTok{(x))}
\NormalTok{p }\OperatorTok{+}\StringTok{ }\KeywordTok{geom_bar}\NormalTok{(}\DataTypeTok{fill =} \StringTok{"grey50"}\NormalTok{)}
\end{Highlighting}
\end{Shaded}

\includegraphics[height=5.25cm]{SappoRoR9_files/figure-beamer/geom_bar-1}

\end{frame}

\begin{frame}[fragile]{stat\_*関数}

統計的変換, 例) \textbf{\texttt{stat\_count}}:
データの頻度分布を計算する。

\begin{Shaded}
\begin{Highlighting}[]
\NormalTok{p }\OperatorTok{+}\StringTok{ }\KeywordTok{stat_count}\NormalTok{(}\DataTypeTok{geom =} \StringTok{"bar"}\NormalTok{, }\DataTypeTok{fill =} \StringTok{"grey50"}\NormalTok{)}
\end{Highlighting}
\end{Shaded}

\includegraphics[height=4cm]{SappoRoR9_files/figure-beamer/stat_count-1}

\begin{itemize}
\tightlist
\item
  たいていの\texttt{geom}には対応する\texttt{stat}関数がある。
\item
  \texttt{geom}ではなくて\texttt{stat}で描画レイヤーを追加することも可能。
\end{itemize}

\end{frame}

\begin{frame}[fragile]{関数のグラフと数式}

\textbf{stat\_function}で関数を指定。 \textbf{\texttt{annotate}}で注釈。
\texttt{parse\ =\ TRUE}で数式を解釈して表示。

\begin{Shaded}
\begin{Highlighting}[]
\KeywordTok{ggplot}\NormalTok{(}\KeywordTok{data.frame}\NormalTok{(}\DataTypeTok{x =} \KeywordTok{seq}\NormalTok{(}\OperatorTok{-}\DecValTok{5}\NormalTok{, }\DecValTok{5}\NormalTok{, }\FloatTok{0.01}\NormalTok{)), }\KeywordTok{aes}\NormalTok{(x)) }\OperatorTok{+}
\StringTok{  }\KeywordTok{stat_function}\NormalTok{(}\DataTypeTok{fun =} \ControlFlowTok{function}\NormalTok{(x) \{}\DecValTok{1} \OperatorTok{/}\StringTok{ }\NormalTok{(}\DecValTok{1} \OperatorTok{+}\StringTok{ }\KeywordTok{exp}\NormalTok{(}\OperatorTok{-}\NormalTok{x))\}) }\OperatorTok{+}
\StringTok{  }\KeywordTok{annotate}\NormalTok{(}\StringTok{"text"}\NormalTok{, }\DataTypeTok{x =} \OperatorTok{-}\DecValTok{5}\NormalTok{, }\DataTypeTok{y =} \FloatTok{0.875}\NormalTok{,}
      \DataTypeTok{label =} \KeywordTok{paste}\NormalTok{(}\StringTok{"italic(y) == frac(1,"}\NormalTok{, }
                    \StringTok{"1 + exp(-(beta[0] + beta[1]*italic(x))))"}\NormalTok{),}
      \DataTypeTok{parse =} \OtherTok{TRUE}\NormalTok{, }\DataTypeTok{size =} \DecValTok{4}\NormalTok{, }\DataTypeTok{hjust =} \DecValTok{0}\NormalTok{)}
\end{Highlighting}
\end{Shaded}

\includegraphics[height=4.5cm]{SappoRoR9_files/figure-beamer/expression-1}

\end{frame}

\begin{frame}[fragile]{地図データ}

@u\_riboさん作のjpndisrictパッケージを利用する。

\begin{Shaded}
\begin{Highlighting}[]
\KeywordTok{library}\NormalTok{(jpndistrict)}

\NormalTok{hokkaido <-}\StringTok{ }\KeywordTok{jpn_pref}\NormalTok{(}\DataTypeTok{pref_code =} \DecValTok{1}\NormalTok{)}
\NormalTok{sapporo <-}\StringTok{ }\KeywordTok{jpn_cities}\NormalTok{(}\DataTypeTok{jis_code =} \KeywordTok{paste0}\NormalTok{(}\StringTok{"0"}\NormalTok{, }\DecValTok{1101}\OperatorTok{:}\DecValTok{1110}\NormalTok{))}
\end{Highlighting}
\end{Shaded}

\end{frame}

\begin{frame}[fragile]{地図データの描画}

ver. 3で追加になった\textbf{geom\_sf}を使う。

\begin{Shaded}
\begin{Highlighting}[]
\KeywordTok{ggplot}\NormalTok{(hokkaido) }\OperatorTok{+}\StringTok{ }\KeywordTok{geom_sf}\NormalTok{(}\KeywordTok{aes}\NormalTok{(}\DataTypeTok{fill =}\NormalTok{ city)) }\OperatorTok{+}
\StringTok{  }\KeywordTok{theme}\NormalTok{(}\DataTypeTok{legend.position =} \StringTok{"none"}\NormalTok{)}
\end{Highlighting}
\end{Shaded}

\includegraphics[height=3.6cm]{SappoRoR9_files/figure-beamer/geom_sf-1}

\end{frame}

\begin{frame}[fragile]

\begin{Shaded}
\begin{Highlighting}[]
\KeywordTok{ggplot}\NormalTok{(sapporo) }\OperatorTok{+}\StringTok{ }\KeywordTok{geom_sf}\NormalTok{(}\KeywordTok{aes}\NormalTok{(}\DataTypeTok{fill =}\NormalTok{ city)) }\OperatorTok{+}
\StringTok{  }\KeywordTok{scale_fill_discrete}\NormalTok{(}\DataTypeTok{name =} \StringTok{"市区"}\NormalTok{, }\DataTypeTok{breaks =}\NormalTok{ sapporo}\OperatorTok{$}\NormalTok{city) }\OperatorTok{+}
\StringTok{  }\KeywordTok{theme_bw}\NormalTok{(}\DataTypeTok{base_family =} \StringTok{"IPAexGothic"}\NormalTok{)}
\end{Highlighting}
\end{Shaded}

\includegraphics[height=3.6cm]{SappoRoR9_files/figure-beamer/geom_sf2-1}

\end{frame}

\section{おわりに}

\begin{frame}{参考リンク}

きょう紹介できたのはごく一部。下のリンクなどを参考に。

\begin{itemize}
\tightlist
\item
  ggplot2(公式ウェブサイト)

  \begin{itemize}
  \tightlist
  \item
    \url{http://ggplot2.tidyverse.org/}
  \end{itemize}
\item
  ggplot2に関する資料(前田和寛 `@kazutan' さん)

  \begin{itemize}
  \tightlist
  \item
    \url{https://kazutan.github.io/kazutanR/ggplot2_links.html}
  \end{itemize}
\item
  ggplot2 --- きれいなグラフを簡単に合理的に(岩嵜航さん)

  \begin{itemize}
  \tightlist
  \item
    \url{https://heavywatal.github.io/rstats/ggplot2.html}
  \end{itemize}
\item
  グラフ描画ggplot2の辞書的まとめ20のコード(MrUnadonさん)

  \begin{itemize}
  \tightlist
  \item
    \url{https://mrunadon.github.io/ggplot2/}
  \end{itemize}
\end{itemize}

\end{frame}

\begin{frame}

\begin{itemize}
\tightlist
\item
  ggplot2逆引(Hiroaki Yutaniさん)

  \begin{itemize}
  \tightlist
  \item
    \url{https://yutannihilation.github.io/ggplot2-gyakubiki/}
  \end{itemize}
\item
  Cookbook for R » Graphs(Winston Changさん)

  \begin{itemize}
  \tightlist
  \item
    \url{http://www.cookbook-r.com/Graphs/}
  \end{itemize}
\item
  Stack Overflowでのggplot2関連の質問

  \begin{itemize}
  \tightlist
  \item
    \url{https://stackoverflow.com/search?q=ggplot2}
  \end{itemize}
\end{itemize}

\end{frame}

\end{document}
